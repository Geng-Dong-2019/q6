%===============================================================================
%  Preamble
%===============================================================================
\documentclass[10pt, oneside, pdftex]{article}
\renewcommand{\thefootnote}{\roman{footnote}}
\usepackage{amsmath, amstext, amsfonts, amssymb}
\usepackage{xcolor}
\definecolor{bl}{rgb}{0.0,0.2,0.6}
\usepackage{sectsty}
\usepackage{url}
\usepackage{fancyvrb}
\usepackage{setspace}
\singlespacing
\usepackage[left=2.54cm,bottom=2.00cm,top=2.00cm,right=2.54cm]{geometry}
\usepackage[compact]{titlesec}
\usepackage[colorinlistoftodos]{todonotes}
%\allsectionsfont{\color{bl}\scshape\selectfont}
\allsectionsfont{\color{bl}}
%\usepackage{chapterbib}
%\includeonly{chapter1}
%\usepackage{pdflscape}
%\usepackage[style=reading, sorting=ynt]{biblatex}


%===============================================================================
% Setup of lstlisting
%===============================================================================
\usepackage{listings}
\usepackage{color}
\definecolor{deepblue}{rgb}{0,0,0.5}
\definecolor{deepred}{rgb}{0.6,0,0}
\definecolor{deepgreen}{rgb}{0,0.5,0}
\definecolor{dkgreen}{rgb}{0,0.6,0}
\definecolor{gray}{rgb}{0.5,0.5,0.5}
\definecolor{mauve}{rgb}{0.58,0,0.82}

\lstset{frame=tb,
language=bash,
aboveskip=3mm,
belowskip=3mm,
showstringspaces=false,
columns=flexible,
basicstyle={\small\ttfamily},
numbers=none,
numberstyle=\tiny\color{gray},
keywordstyle=\color{blue},
commentstyle=\color{dkgreen},
stringstyle=\color{mauve},
breaklines=true,
breakatwhitespace=true,
tabsize=3}


%===============================================================================
%  Fonts in Document
%===============================================================================
%\usepackage{times}
%\usepackage{pslatex}
%\usepackage{newcent}
%\usepackage{palatcm}
%\usepackage{palatino}
%\usepackage[T1]{fontenc}
\usepackage[scaled]{helvet}
\renewcommand*\familydefault{\sfdefault}
%\renewcommand*{\encodingdefault}{T1}
% For more options go to: http://www.tug.dk/FontCatalogue/


%===============================================================================
% Definitions
%===============================================================================
% Define a new command that prints the title only
% Begin definition
\makeatletter
\def\printtitle{
{\color{bl} \centering \Huge \textbf{\@title}\par}} % Typesetting
\makeatother
% End definition

\title{Q\\
    \LARGE Version 5.7 \\
    \LARGE \vspace*{-10pt}Developers Manual\vspace*{10pt}}


% Define a new command that prints the author(s) only
% Begin definition
\makeatletter
\def\printauthor{                        % Define command: \printauthor
{\centering \small \@author}}            % Typesetting
\makeatother
% End definition

\author{
  \vspace{10pt}
  Mauricio Esguerra \\
  \vspace{4pt}
  mauricio.esguerra@gmail.com \\
  \vspace{20pt}
  }

% Custom headers and footers
\usepackage{fancyhdr}
\pagestyle{fancy}                    % Enabling the custom headers/footers
\usepackage{lastpage}
% Header (empty)
\lhead{}
\chead{}
\rhead{}
\lfoot{\footnotesize Q Development - \texttt{qdyn.org}}
\cfoot{}
\rfoot{\footnotesize page \thepage\ }%of \pageref{LastPage}}    % "Page 1 of 2"
\renewcommand{\headrulewidth}{0.0pt}
\renewcommand{\footrulewidth}{0.4pt}

% Change the abstract environment
\usepackage[runin]{abstract}             % runin option for a run-in title
\setlength\absleftindent{30pt}           % left margin
\setlength\absrightindent{30pt}          % right margin
\abslabeldelim{\quad}                    %
\setlength{\abstitleskip}{-10pt}
\renewcommand{\abstractname}{}
\renewcommand{\abstracttextfont}{\color{bl} \small \slshape}    % slanted text

%===============================================================================
% Start document
%===============================================================================
\begin{document}
\printtitle
\printauthor
\tableofcontents
%\bibliographystyle{plainurl}
\bibliographystyle{nar}
\vspace{20pt}
\section{Introduction}
This is  the developers manual  for \textbf{Q}, the free  energy from
molecular dynamics program.\\
\noindent \textbf{Q}  is entirely written in  modern \textbf{FORTRAN}.
Its  initial   development  was   in  \textbf{FORTRAN}  90,   but  the
incorporation    of   object    oriented    paradigms   into    modern
\textbf{FORTRAN} have  seen the code  evolve to take advantage  of the
evolution  of the  language.   Currently  the program  incorporporates
elements of  the \textbf{FORTRAN}  2008 standard.  For parallelization
\textbf{Q}  uses  the  MPI2  standard  and  follows  a  point-to-point
communication  approach. This  is advantageous  in that  it can  avoid
overhead which often happens  in one-sided communication, specially if
one does not  develop according to the system specifics,  like type of
processors, OS, number of processors, etc.


\subsection{Makefile}
\label{makefile}
The  makefile follows  closely the  \textbf{GNU} \textbf{make}  manual
(\url{http://www.gnu.org/software/make/manual/make.html}).         Any
modifications to the makefile should  make sure that user-space is not
broken,  that  is,  \textbf{Q}  should  compile on  any  of  the  main
operating  systems and  avoid as  much as  possible hardware  specific
compiler options.\\

\subsection{Cluster Architectures}
\label{clusterarchitectures}


\subsection{Debugging}
\label{debugging}


\subsection{Profiling}
\label{profiling}

\subsection{Best Practices}
\label{bestpractices}

\url{http://www.fortran90.org/src/best-practices.html}


\section{ChangeLog}
\label{changelog}


\section{Release Schedule}
\label{schedule}


\section{qprep}
\label{qprep}
\textbf{qprep}  is the  utility which  allows the  translation of  pdb
files into a format which the dynamics engine can understand. The main
file  generated  by \textbf{qprep}  is  refered  to as  the  topology,
generally  given the  \textit{.top} extension.  The following  command
passes an input file to qprep.

\lstset{language=sh, frame=single}
%\begin{Verbatim}[numbers=left]
\fvset{frame=single, fontfamily=courier, fontsize=\small}
%\fvset{frame=single, fontfamily=courier}
\begin{Verbatim}
qprep < generate.inp >& generate.log &
\end{Verbatim}

\subsection{qprep.f90}

\textbf{qprep.f90} is the  main program which acts as  an interface to
the main  code which resides  mainly in the  \textbf{prep.f90} module.
It's composed of six subroutines.\\

\textbf{Modules: }\\
\textit{iso\_fortran\_env}\\
\textit{prep}\\
\textit{avetr}\\


\textbf{Subroutines: }\\


\subsubsection{version} 
\subsubsection{prep}
\textbf{Modules: }\\
\textit{}

%%qprep.o prefs.o prep.o avetr.o nrgy.o


\begin{Verbatim}
subroutine prep_startup
subroutine allocate_for_pdb(atoms, residues, molecules)
subroutine prep_shutdown
subroutine clearlib
subroutine check_alloc(message)
subroutine addbond
subroutine clearbond
subroutine angle_ene(emax, nlarge, av_ene)
subroutine bond_ene(emax, nlarge, av_ene)
subroutine changeimp
subroutine checkangs
subroutine checkbonds
subroutine checkimps
subroutine checktors
subroutine xlink
subroutine impr_ene(emax, nlarge, av_ene, mode)
subroutine listres
subroutine listseq
subroutine make14list
subroutine makeangles
subroutine make_solute_bonds
subroutine makebonds
subroutine set_bondcodes
subroutine makeexlist
subroutine makehyds
subroutine makeimps
subroutine imp_params
subroutine makeimps_explicit
subroutine maketop
subroutine set_default_mask
subroutine makeconn
subroutine maketors
subroutine prompt(outtxt)
subroutine oldreadlib(filnam)
subroutine readlib(file)
subroutine check_overload(resnam)
subroutine oldreadparm(flag)
subroutine readff
subroutine readparm(filnam)
subroutine clearpdb
subroutine cleartop
subroutine readpdb()
subroutine readtop
subroutine readx
subroutine readnext
subroutine readframe
subroutine trajectory
subroutine modify_mask
subroutine set_cgp
subroutine set_crg
subroutine set_iac
subroutine set_solvent_type
subroutine define_boundary_condition
subroutine solvate
subroutine solvate_box
subroutine solvate_box_grid
subroutine solvate_box_file
subroutine solvate_sphere
subroutine solvate_sphere_grid
subroutine solvate_sphere_file(shift)
subroutine solvate_restart
subroutine add_solvent_to_topology(waters_in_sphere, max_waters,
make_hydrogens, pack)
subroutine grow_arrays_for_solvent(nmore, atoms_per_molecule)
subroutine tors_ene(emax, nlarge, av_ene)
subroutine writepdb
subroutine writemol2
subroutine writetop
subroutine listprefs
subroutine set
subroutine make_shell2
\end{Verbatim}
 
\subsubsection{avetr} 



\section{qdum}
\label{qdum}


\section{qdyn}
\label{qdyn}
\textbf{qdyn} is compiled using the following explicit rules:
\begin{Verbatim}
-ffree-line-length-none
-fcray-pointer
-fall-intrinsics
-std=legacy
-Wall
-Wtabs
-fstack-protector
-O3
-DG95=1 -c
\end{Verbatim}

\textbf{Development} flags recommended from fortran90.org:\\
-Wall -Wextra -Wimplicit-interface -fPIC -fmax-errors=1 -g -fcheck=all -fbacktrace\\

\textbf{Production} flags recommended from fortran90.org:\\
-Wall -Wextra -Wimplicit-interface -fPIC -Werror -fmax-errors=1 -O3 -march=native -ffast-math -funroll-loops\\

sizes: Specifies data storage for all Q programs.\\
Modules used= NONE

nrgy: Input/Output for energy data and energy file.\\
Modules used= sizes\\
Subroutines:\\
  nrgy\_startup = empty\\
  put\_ene =

Functions:

mpiglob: Declares global variables for MPI parallelization of qdyn
Modules used = nrgy
Subroutines:

Functions:


misc:\\
Modules used = sizes\\
Subroutines:

Functions:


prmfile: Data files parser.
Modules used = misc
Subroutines:
  prmfile\_startup = EMPTY ? WTF


Functions:


index:
Modules used = NONE


topo:
Modules used = sizes, misc
Subroutines:
  topo\_startup = EMPTY ? WTF


Functions:


qatom:
Modules used = sizes, nrgy, misc, prmfile, indexer, topo


mask:
Modules used = topo


trj:
Modules used = atom\_mask, misc




-DG95=1 -cpp -c

md: This is the main molecular dynamics \textit{module}.
Modules used =


qdyn: This file contains the main \textbf{qdyn} \textit{program} code.
Modules used =



\section{qfep}
\label{qfep}


\section{qcalc}
\label{qcalc}


\section{Tricks}
\label{tricks}

A very convenient trick for cleaning up the code is:

emacs --batch md.f90 -f mark-whole-buffer -f f90-indent-subprogram -f save-buffer


\section{Links}
\label{links}
The following  is a  collection of reference  links useful  for fotran
programers.


\noindent - The description of the GNU Compiler Collection (gcc) version 5.1.0\\
\url{https://gcc.gnu.org/onlinedocs/gcc-5.1.0/gfortran.pdf}


\subsection{Doxygen}
\label{doxygen}
To  automatically document  fortran  code the  only current  available
option is doxygen.
To  use doxygen  a  configuration  file is  necessary.  This has  been
downloaded from:\\

\url{https://modelingguru.nasa.gov/docs/DOC-1811}

And modified to our needs.

Just doing:

\begin{lstlisting}
doxygen DoxygenConfigFortran
\end{lstlisting}


Generates html code which we make available at:

doxygen.qdyn.org

A mainfile document is included in the file maintext.txt


\subsection{Fortran codes available online}
These projects are  a good resource to check-out  development and good
practices on coding. Even though FORTRAN does not enforce syntax rules
as  Python  does,  the user  should  use  them  so  that the  code  is
readable. An analog of the  sytax rules for Python, frequently refered
to as PEP8, would be very useful for FORTRAN programers. \\

Jason Blevins. Benchmarking ansi C and FORTRAN.\\
\url{http://jblevins.org/git/scicomp.git/tree/}\\

Gyro-Kinetics at Warwick. A program for turbulence study in plasmas.\\
\url{https://bitbucket.org/gkw/gkw/wiki/Home}\\

OpenMC. A Monte Carlo code.\\
\url{https://github.com/mit-crpg/openmc}\\

A collection of generic Fortran routines.\\
\url{https://github.com/astrofrog/fortranlib}\\


\bibliography{biblio}
\end{document}
