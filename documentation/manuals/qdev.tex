%===============================================================================
%  Preamble
%===============================================================================
\documentclass[10pt, oneside, pdftex]{article}
\renewcommand{\thefootnote}{\roman{footnote}}
\usepackage{amsmath, amstext, amsfonts, amssymb}
\usepackage{xcolor}
\definecolor{bl}{rgb}{0.0,0.2,0.6}
\usepackage{sectsty}
\usepackage{url}
\usepackage{fancyvrb}
\usepackage{setspace}
\singlespacing
\usepackage[left=2.54cm,bottom=2.00cm,top=2.00cm,right=2.54cm]{geometry}
\usepackage[compact]{titlesec}
%\allsectionsfont{\color{bl}\scshape\selectfont}
\allsectionsfont{\color{bl}}
%\usepackage{chapterbib}
%\includeonly{chapter1}
%\usepackage{pdflscape}
%\usepackage[style=reading, sorting=ynt]{biblatex}


%===============================================================================
% Setup of lstlisting
%===============================================================================
\usepackage{listings}
\usepackage{color}
\definecolor{deepblue}{rgb}{0,0,0.5}
\definecolor{deepred}{rgb}{0.6,0,0}
\definecolor{deepgreen}{rgb}{0,0.5,0}
\definecolor{dkgreen}{rgb}{0,0.6,0}
\definecolor{gray}{rgb}{0.5,0.5,0.5}
\definecolor{mauve}{rgb}{0.58,0,0.82}

\lstset{frame=tb, 
language=bash, 
aboveskip=3mm,
belowskip=3mm, 
showstringspaces=false, 
columns=flexible,
basicstyle={\small\ttfamily}, 
numbers=none,
numberstyle=\tiny\color{gray}, 
keywordstyle=\color{blue},
commentstyle=\color{dkgreen},
stringstyle=\color{mauve},
breaklines=true,
breakatwhitespace=true,
tabsize=3}



%===============================================================================
%  Fonts in Document
%===============================================================================
%\usepackage{times}
%\usepackage{pslatex}
%\usepackage{newcent}
%\usepackage{palatcm}
%\usepackage{palatino}
%\usepackage[T1]{fontenc}
\usepackage[scaled]{helvet}
\renewcommand*\familydefault{\sfdefault}
%\renewcommand*{\encodingdefault}{T1}
% For more options go to: http://www.tug.dk/FontCatalogue/
\usepackage[colorinlistoftodos]{todonotes}


%===============================================================================
% Definitions
%===============================================================================
% Define a new command that prints the title only
\makeatletter                                           % Begin definition
\def\printtitle{                                        % Define command: \printtitle
{\color{bl} \centering \huge \sc \textbf{\@title}\par}} % Typesetting
\makeatother                                            % End definition

\title{Q DEVELOPERS MANUAL} %\\
%    \large \vspace*{-10pt}A Developers Manual for Q.\vspace*{10pt}}

% Define a new command that prints the author(s) only
\makeatletter                            % Begin definition
\def\printauthor{                        % Define command: \printauthor
{\centering \small \@author}}            % Typesetting
\makeatother                             % End definition

\author{
    Mauricio Esguerra \\
    mauricio.esguerra@gmail.com \\
    \vspace{20pt}
    }

% Custom headers and footers
\usepackage{fancyhdr}
\pagestyle{fancy}                    % Enabling the custom headers/footers
\usepackage{lastpage}
% Header (empty)
\lhead{}
\chead{}
\rhead{}
% Footer (you may change this to your own needs)
\lfoot{\footnotesize \texttt{mesguerra.org} - Q Development}
\cfoot{}
\rfoot{\footnotesize page \thepage\ }%of \pageref{LastPage}}    % "Page 1 of 2"
\renewcommand{\headrulewidth}{0.0pt}
\renewcommand{\footrulewidth}{0.4pt}

% Change the abstract environment
\usepackage[runin]{abstract}             % runin option for a run-in title
\setlength\absleftindent{30pt}           % left margin
\setlength\absrightindent{30pt}          % right margin
\abslabeldelim{\quad}                    %
\setlength{\abstitleskip}{-10pt}
\renewcommand{\abstractname}{}
\renewcommand{\abstracttextfont}{\color{bl} \small \slshape}    % slanted text

%===============================================================================
% Start document
%===============================================================================
\begin{document}
\printtitle
\printauthor
\tableofcontents
%\bibliographystyle{plainurl}
\bibliographystyle{nar}
\section{Introduction}
This is the developers manual for \textbf{Q}. the free enegies from molecular dynamics
program.
Q is entirely writen in Fortran. It's initial development was in Fortran 90, but
the incorporation of object oriented paradigms into modern Fortran have seen
the code evolve to take advantage of the evolution of the language.

A very convenient trick fro cleaning up the code is:

emacs --batch md.f90 -f mark-whole-buffer -f f90-indent-subprogram -f save-buffer

\label{makefile}
\subsection{makefile}

\label{clusterarchitectures}
\subsection{Cluster Architectures}

\label{debugging}
\subsection{Debugging}

\label{profiling}
\subsection{profiling}

\label{bestpractices}
\subsection{Best Practices}

\url{http://www.fortran90.org/src/best-practices.html}

\label{changelog}
\section{ChangeLog}

\label{schedule}
\section{Release Schedule}

\label{qprep}
\section{qprep}
\textbf{qprep} is the utility which allows the translation of pdb files into
a format which the dynamics engine can understand. The main file generated
by \textbf{qprep} is refered to as the topology.

\lstset{language=sh, frame=single}
%\begin{Verbatim}[numbers=left]
\fvset{frame=single, fontfamily=courier, fontsize=\small}
%\fvset{frame=single, fontfamily=courier}
\begin{Verbatim}
bash-3.1\$ qprep generate.inp >& generate.out &
\end{Verbatim}


\begin{Verbatim}
bash-3.1\$ pwd
/home/username
bash-3.1\$ cd Desktop
bash-3.1\$ ls
Trash
\end{Verbatim}


\section{qdyn}
\textbf{qdyn} is compiled using the following explicit rules:
-ffree-line-length-none
-fcray-pointer
-fall-intrinsics 
-std=legacy 
-Wall 
-Wtabs 
-fstack-protector 
-O3 
-DG95=1 -c 

Development flags recommended from fortran90.org:
-Wall -Wextra -Wimplicit-interface -fPIC -fmax-errors=1 -g -fcheck=all -fbacktrace

Production flags recommended from fortran90.org:
-Wall -Wextra -Wimplicit-interface -fPIC -Werror -fmax-errors=1 -O3 -march=native -ffast-math -funroll-loops


sizes: Does this.
Modules used= NONE


nrgy: Does this.
Modules used= sizes
Subroutines: 
  nrgy\_startup = empty
  put\_ene = 

Functions:


mpiglob: Declares global variables for MPI parallelization of qdyn
Modules used = nrgy
Subroutines:

Functions:


misc:
Modules used = sizes
Subroutines:

Functions:


prmfile: Data files parser.
Modules used = misc
Subroutines: 
  prmfile\_startup = EMPTY ? WTF


Functions:


index:
Modules used = NONE


topo:
Modules used = sizes, misc
Subroutines: 
  topo\_startup = EMPTY ? WTF


Functions:


qatom:
Modules used = sizes, nrgy, misc, prmfile, indexer, topo


mask:
Modules used = topo


trj:
Modules used = atom\_mask, misc




-DG95=1 -cpp -c

md: This is the main molecular dynamics \textit{module}.
Modules used = 


qdyn: This file contains the main \textbf{qdyn} \textit{program} code.
Modules used = 




\subsection{subsection 1}

Make sure to have the location of the binaries on your path.\\

\noindent
Make sure to modify rungms to have the path to the gamess binary and
also the location of scratch.\\

\noindent
Make a  symbolic link to the  mac executable which comes  with a funky
name.

\section{qfep}


\section{qcalc}

\section{Links}
The following  is a  collection of reference  links useful  for fotran
programers.

The description of the GNU Compiler Collection (gcc) version 5.1.0
\url{https://gcc.gnu.org/onlinedocs/gcc-5.1.0/gfortran.pdf}


\subsection{Doxygen}
To  automatically document  fortran  code the  only current  available
option is doxygen.
To  use doxygen  a  configuration  file is  necessary.  This has  been
downloaded from:\\

\url{https://modelingguru.nasa.gov/docs/DOC-1811}

And modified to our needs.

Just doing:

\begin{lstlisting}
doxygen DoxygenConfigFortran
\end{lstlisting}


Generates html code which we make available at:

doxygen.qdyn.org

A mainfile document is included in the file maintext.txt



\subsection{Fortran codes available online}

These  projects  are  a  good resource  to  check-out  development  and
practices on coding.

Gyro-Kinetics at Warwick. A program for turbulence study in plasmas.\\
\url{https://bitbucket.org/gkw/gkw/wiki/Home}

OpenMC. A Monte Carlo code.\\
\url{https://github.com/mit-crpg/openmc}


A collection of generic Fortran routines.
\url{https://github.com/astrofrog/fortranlib}


\bibliography{biblio}
\end{document}
